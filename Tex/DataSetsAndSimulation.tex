\section{Data and Monte Carlo samples}
\label{sec:dataset}
This analysis uses 139~fb$^{-1}$ of data collected from proton-proton collision recorded by the ATLAS 
detector at $\sqrt{s}=13$ TeV during 2015-2018. The data set has been collected with a 
bunch crossing of 25 ns, IBL on, and verifying data quality cuts namely which must be in the
recommended Good Run List.
%\footnote{For 2015 dataset:
%\textit{data15\_13TeV.periodAllYear\_DetStatus-v79-repro20-02\_DQDefects-00-02-02\_PHYS\_StandardGRL\_All\_Good\_25ns.xml};
%for 2016 dataset: 
%\textit{data16\_13TeV.periodAllYear\_DetStatus-v88-pro20-21\_DQDefects-00-02-04\_PHYS\_StandardGRL\_All\_Good\_25ns.xml} \\}.  

The analysis uses data being prepared with \verb|xAOD| format and further produced to \verb|DxAOD| 
format using \verb|HIGG8D1| derivation framework. This \verb|xAOD| to \verb|DxAOD| derivation 
provides a reduction specifically for the \tth events with multileptons in the final states. 
The total size of data set has been reduced to 3.6 \% for simulated \ttbar 
events and 0.1\% for collision data set. The size reduction is the result of applying 
smart slimming (remove un-needed variables), thinning (remove entire objects from events) and 
additional skimming on both collision dataset and MC samples. The skimming in \verb|HIGG8D1| derivations consists of removing an event if 
it does not fulfil the following selection: 
\begin{itemize}
  \item at least two light leptons passing loose identification criteria with leading lepton $p_{T}>$ 15 GeV and subleading
  lepton $p_{T}>$ 5GeV, within $|\eta|<$ 2.6
  \item at least one light lepton passing loose identification criteria with $p_{T}>$ 15 GeV and $|\eta|<$ 2.6, and at least
  two hadronic taus. The tau lepton has to pass \verb|JetBDTSigLoose| requirement with $p_{T}>$ 15 GeV, its charge must be
  one and it must have one or three associated tracks.
\end{itemize}

\subsection{Monte Carlo samples}
\label{subsec:ms}

\begin{table}
\begin{center}
{\small
\setlength\tabcolsep{1.5pt}
\begin{tabular}{llllll}
\hline\hline
Process & Generator & Parton Shower & PDF & Tune  \\
& (alternative) & (alternative) & & \\
\hline
\ttH & \textsc{Powheg-BOX} \cite{powhegtt}  & \textsc{Pythia} 8\ & NNPDF 3.0 NLO \cite{Ball:2014uwa}/ & A14 \\
     &                                         &                                       & NNPDF 2.3 LO \cite{Ball:2012cx} \\
     & (-) & (\textsc{Herwig++}) &  \\
$tHqb$ & \textsc{MG5\_aMC} & \textsc{Pythia} 8 & CT10 \cite{ct10} & A14  & \\
$tHW$ & \textsc{MG5\_aMC} & \textsc{Herwig++}  & CT10 & UE-EE-5   \cite{Seymour:2013qka}   \\
& & & /CTEQ6L1 \cite{cteq6l1,cteq6}  \\
%$\ttbar W$ & \textsc{MG5\_aMC} & \textsc{Pythia} 8 & NNPDF 3.0 NLO & A14   \\
%& & & /2.3 LO \\
$\ttbar W$ & \textsc{Sherpa 2.2.1}~\cite{sherpa} & \textsc{Sherpa 2.2.1}  & NNPDF 3.0 NNLO  & \textsc{Sherpa} default   \\
& (\textsc{MG5\_aMC}) & (\textsc{Pythia} 8) &  \\
$\ttbar (Z/\gamma^*)$ & \textsc{MG5\_aMC} & \textsc{Pythia} 8 & NNPDF 3.0 NLO & A14  \\
&&& /2.3 LO \\
& (\textsc{Sherpa}) & (\textsc{Sherpa}) &  \\
$t (Z/\gamma^*)$ & \textsc{MG5\_aMC} & \textsc{Pythia} 8  & CTEQ6L1 & Perugia2012 \cite{perugia}  \\
$t W (Z/\gamma^*)$ & \textsc{MG5\_aMC} & \textsc{Pythia} 8 & NNPDF 2.3 LO  & A14  \\
$t\bar t t$, $t\bar t t\bar t$ & \textsc{MG5\_aMC} & \textsc{Pythia} 8 & NNPDF 2.3 LO & A14 \\
$t\bar t W^+ W^-$ & \textsc{MG5\_aMC} & \textsc{Pythia} 8 & NNPDF 2.3 LO & A14  \\
$\ttbar$ & \textsc{Powheg-BOX} \cite{powhegtt} & \textsc{Pythia} 8 & CT10/CTEQ6L1 & Perugia2012  \\
$\ttbar\gamma$ & \textsc{MG5\_aMC} & \textsc{Pythia} 8 & NNPDF 2.3 LO & A14  \\
$s$-, $t$-channel, & \textsc{Powheg-BOX} \cite{powhegstp,powhegstp2} & \textsc{Pythia} 6 & CT10 & Perugia2012   \\
 $Wt$ single top & & & /CTEQ6L1 \\
$VV$, $qqVV$, & \textsc{Sherpa} 2.2.2 \cite{sherpa} & \textsc{Sherpa} & NNPDF 3.0 NNLO & \textsc{Sherpa} default  \\
$VVV$ & & & \\
$Z \to \ell^+\ell^-$ & \textsc{Sherpa} 2.2 & \textsc{Sherpa} & NNPDF 3.0 NLO & \textsc{Sherpa} default \\
%$W \to \ell\nu$ & \textsc{Sherpa} & \textsc{Sherpa} & CT10 & \textsc{Sherpa} default \\
\hline\hline
\end{tabular}
}
\caption{\label{tab:mcconfig} The configurations used for event generation of signal and background processes. 
If only one parton distribution function (PDF) is shown, the same one is used for both the matrix element (ME) and parton shower generators; 
if two are shown, the first is used for the matrix element calculation and the second for the parton shower.  ``V'' refers to production of 
an electroweak boson ($W$ or $Z/\gamma^*$).  ``Tune'' refers to the underlying-event tune of the parton shower generator. ``\textsc{MG5\_aMC}'' 
refers to \textsc{MadGraph5\_aMC@NLO} 2.2.1~\cite{Alwall:2014hca}; ``\textsc{Pythia} 6'' refers to version 6.427~\cite{Pythia6}; ``\textsc{Pythia} 8'' 
refers to version 8.2~\cite{Pythia8}; ``\textsc{Herwig++}'' refers to version 2.7~\cite{Bahr:2008pv}.  Samples using \textsc{Pythia} 6 or \textsc{Pythia} 8  
have heavy flavour hadron decays modelled by \textsc{EvtGen} 1.2.0~\cite{Lange:2001uf}.  All samples include leading-logarithm photon emission, either modelled 
by the parton shower generator or by \textsc{PHOTOS}~\cite{Golonka:2005pn}.}
\end{center}
\end{table}

